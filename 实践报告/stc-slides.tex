\documentclass[aspectratio=169]{beamer}
\usepackage{geometry}

\usetheme{metropolis}           % Use metropolis theme
\usefonttheme[onlymath]{serif}

\usepackage{xeCJK}
%\usepackage{CJKutf8}
\usepackage{graphicx}
\usepackage{amsmath}
\usepackage{multicol}
\usepackage[font=small,labelfont=bf]{caption}
\DeclareCaptionFont{mysize}{\fontsize{7}{9.6}\selectfont}
\captionsetup{font=mysize}

\usepackage{textpos}
\usepackage{adjustbox}
\usepackage{pdfpages}

\newcommand\Fontvi{\fontsize{8}{7.2}\selectfont}

\title{实习实践报告:短文本对话}
\author{报告人:林连升 \\ 校内导师:叶东毅 \ 教授 \\ 企业导师:关胤}
\institute{实习单位:福建天晴在线互动科技有限公司}
\date{}

%\logo{\includegraphics[height=1.2cm,width=1.35cm]{logo}}

\titlegraphic{\hspace*{11.2cm} \includegraphics[height=3cm,width=3.38cm]{logo}
}

\begin{document}
  \maketitle

  \addtobeamertemplate{frametitle}{}{%
  \begin{textblock*}{100mm}(.96\textwidth,-1.1cm)
  \includegraphics[height=1.2cm,width=1.35cm]{logo}
  \end{textblock*}}

    \begin{frame}{短文本对话}
      
      \begin{figure}
      \includegraphics[width=10cm,height=5.09cm]{stc2-ir-task.png}
      \caption{基于检索的方法}
      \end{figure}

    \end{frame}

    \begin{frame}{相关性评估}
      评估检索出的评论主要遵循以下四个准则:
      \begin{enumerate}
        \item Fluent
        \item Coherent: logically and topically relevant 
        \item Self-sufficient
        \item Substantial
      \end{enumerate}
      If either (1) or (2) is untrue, the retrieved comment should be labeled "L0"; if either (3) or (4) is untrue, the label should be "L1"; otherwise, the label is "L2".
    \end{frame}

        \begin{frame}{Labels}
          \begin{center}
            \begin{figure}
            \includegraphics[width=4in,height=2.25in]{stc-labels.png}
            \caption{一条微博以及人工标注的五条候选评论}
            \end{figure}
          \end{center}
        \end{frame}

    \begin{frame}{System Architecture}
      \begin{figure}
      \includegraphics[width=14cm,height=4.63cm]{stc-flow-big.png}
      \caption{System Architecture}
      \end{figure}
    \end{frame}

    \begin{frame}{文本预处理}
      \begin{itemize}
        \item 繁简转换
        \item 全角转半角
        \item 分词 (pynlpir, PKU标准)
        \item token替换(<\_NUM>, <\_TIME>, <\_URL>)
        \item 过滤无意义的词和特殊符号
      \end{itemize}
    \end{frame}

    {
    \setbeamercolor{background canvas}{bg=}
    \includepdf[pages=1]{test.pdf}
    }

    \begin{frame}{相似度特征}
      \begin{itemize}
        \item TF-IDF
        \item LSA (Latent Semantic Analysis)
        \item LDA (Latent Dirichlet Allocation)
        \item Word2Vec (skip-gram)
        \item \textbf{LSTM-Sen2Vec }
      \end{itemize}

      我们将每条微博和它的所有评论合并成一个文档,然后对这些文档训练LSA和LDA模型。其他模型是以句子作为输入。
    \end{frame}

    \begin{frame}{LSTM}
      \begin{columns}
      \begin{column}[t]{0.5\textwidth}
        \begin{equation}
           f_t = \sigma(W_f \cdot [h_{t-1}, x_t] + b_f)
        \end{equation}
        \begin{equation}
           i_t = \sigma(W_i \cdot [h_{t-1}, x_t] + b_i)
        \end{equation}
        \begin{equation}
           \tilde{C}_t = tanh(W_C \cdot [h_{t-1}, x_t] + b_C) 
        \end{equation}
        \begin{equation}
           C_t = f_t * C_{t-1} + i_t * \tilde{C}_t
        \end{equation}
        \begin{equation}
           o_t = \sigma(W_o \cdot [h_{t-1}, x_t] + b_o)
        \end{equation}
        \begin{equation}
           h_t = o_t * tanh(C_t)
        \end{equation}
      \end{column}

      \begin{column}[t]{0.5\textwidth}
        \begin{figure}
        \includegraphics[height=1.5in, width=2.44in]{lstm2.png}
        \caption{LSTM 单元}
        \end{figure}
      \end{column}

      \end{columns}
      \vspace*{2em}

      \tiny
      Mikolov, Toma's. Statistical Language Models Based on Neural Networks. Ph.D. thesis, Brno University of Technology.(2012)

      Zaremba, Wojciech, I. Sutskever, and O. Vinyals. Recurrent Neural Network Regularization. Eprint Arxiv (2014).
    \end{frame}

    \begin{frame}{Attention weight}
      \begin{columns}
      \begin{column}[t]{0.5\textwidth}
        \begin{figure}
        \includegraphics[width=6.4cm,height=4.8cm]{forward.png}
        \caption{Unidirectional weight distribution}
        \end{figure}
      \end{column}

      \begin{column}[t]{0.5\textwidth}
        \begin{figure}
        \includegraphics[width=6.4cm,height=4.8cm]{bidirectional.png}
        \caption{bidirectional weight distribution}
        \end{figure}
      \end{column}

      \end{columns}
    \end{frame}

    \begin{frame}{LSTM-Sen2Vec}
      \begin{columns}
      \begin{column}[t]{0.5\textwidth}
        \begin{figure}
        \includegraphics[width=6.7cm,height=3.2cm]{unilstm.png}
        \caption{The Unidirectional LSTM}
        \end{figure}
      \end{column}

      \begin{column}[t]{0.5\textwidth}
        \begin{figure}
        \includegraphics[width=6cm,height=3.2cm]{bilstm1.png}
        \caption{The Traditional Bidirectional LSTM}
        \end{figure}
      \end{column}

      \end{columns}
    \end{frame}

    \begin{frame}{LSTM-Sen2Vec}
      \begin{figure}
        \includegraphics[width=12cm,height=6cm]{bilstm2.png}
        \caption{The Modified Bidirectional LSTM}
        \end{figure}
    \end{frame}

    \begin{frame}{生成候选集}
      \begin{itemize}
      \item 相似的微博
      \begin{equation}
         Score_{q,p}^1(q, p) = Sim_{LDA}(q, p) * Sim_{W2V}(q, p) * Sim_{LSTM}(q, p)
      \end{equation}
      \begin{equation}
         Score_{q,p}^2(q, p) = Sim_{LSA}(q, p) * Sim_{W2V}(q, p) * Sim_{LSTM}(q, p)
      \end{equation}
      \item 候选评论
      \begin{equation}
         Score_{q,c}^1(q, c) = Sim_{LSA}(q, c) * Sim_{W2V}(q, c)
      \end{equation}
      \begin{equation}
         Score_{q,c}^2(q, c) = Sim_{LDA}(q, c) * Sim_{W2V}(q, c)
      \end{equation}
    \end{itemize}
    \end{frame}

    \begin{frame}{排序}
      \begin{itemize}
        \item TextRank (Words as vertices)
        \item Pattern-IDF
        \item Pattern-IDF + TextRank (Sentences as vertices)
      \end{itemize}
    \end{frame}

    \begin{frame}{TextRank - 一个基于图的排序算法}
      设$G = (V; E)$为一个由点集V和边集E构成的无向图,其中$E$ 是 $V \times V$的子集. 对于一个给定的点$V_i$, 设$link(V_i)$为与其相连的点的集合. 则一个点$V_i$的分数定义如下:
      \begin{equation}
        WS(V_i) = (1 - d) + d * \sum_{j \in link(V_i)}{w_{ij} * WS({V_j})}
      \end{equation}
      其中$d$是一个damping factor\footnotemark,通常设为0.85.

      \footnotetext{Brin, Sergey, and L. Page. The anatomy of a large-scale hypertextual Web search engine. International Conference on World Wide Web Elsevier Science Publishers B. V. 1998:107-117.}
    \end{frame}

    \begin{frame}{TextRank - Vertices and Edges}
      \begin{itemize}
        \item Vertices: 候选评论中的每个词
        \item Edges: 共现关系
        \item Weighted by: word2vec 相似度和共现次数
      \end{itemize}
    \end{frame}

    \begin{frame}{TextRank - Calculate Iteratively}
      对于$N$个候选评论, $k$个词语, 我们构建一个 $ k \times k $ 的矩阵 $M$. 其中$M_{ij} = cnt * sim(D_i, D_j)$. 然后我们迭代计算
      \[
      R(t+1) = 
      \begin{bmatrix}
          (1-d)/k       \\
          (1-d)/k       \\
          \hdotsfor{1} \\
          (1-d)/k       
      \end{bmatrix}
      +d
      \begin{bmatrix}
          M_{11} & M_{12} & M_{13} & \dots  & M_{1k} \\
          M_{21} & M_{22} & M_{23} & \dots  & M_{2k} \\
          \vdots & \vdots & \vdots & \ddots & \vdots \\
          M_{k1} & M_{k2} & M_{k3} & \dots  & M_{kk}
      \end{bmatrix}
      R(t)
      \]
      %$R(0) = [IDF(D_0) \ IDF(D_1) \ ... \ IDF(D_{k-1})]^T$ \newline
      Stop when $|R(t+1)-R(t)|<\epsilon$, $\epsilon = 10^{-7}$. \\
      这里, $cnt$ 表示$D_i$ 和 $D_j$在一个句子中共现的次数. \\
    \end{frame}

    \begin{frame}{TextRank - Ranking}
      既然我们得到候选评论中每个词 $D_i$的分数$R(D_i)$, 那么每个候选评论$c$的分数可以按下列式子计算:
      \begin{equation}
        Rank_{TextRank}(c) = \frac{\sum_{D_i \in c}{R(D_i)}}{len(c)} 
      \end{equation}
      这里, $len(c)$ 表示候选评论中词语的个数.
    \end{frame}

    \begin{frame}{Pattern-IDF}
      对于微博中的一个词语$D_j$和相应评论中的一个词语$D_i$,我们定义($D_j$,$D_i$)为一个pattern.

      受到TF-IDF的启发, 我们定义Pattern-IDF为:
      \begin{equation}
        PI(D_i|D_j) = 1 / \log_2{\frac{count_c(D_i) * count_p(D_j)}{count_{pair}(D_i, D_j)}}
      \end{equation}
      这里,$count_c$表示词语出现在评论中的次数, $count_p$ 表示词语出现在微博中的次数, $count_{pair}$表示($D_j$,$D_i$)的个数. $count_{pair}(D_i, D_j)$小于3的PI将被移除.

    \end{frame}

    \begin{frame}{Pattern-IDF}
      Let $X = \frac{count_c(D_i) * count_p(D_j)}{count_{pair}(D_i, D_j)}$, then $X \in [1, \infty)$.
      \begin{columns}[T] % align columns
      \begin{column}{.48\textwidth}
      
      \begin{figure}
      \includegraphics[width=6.4cm,height=4.8cm]{pi1.png}
      \caption{log(X)}
      \end{figure}
      
      \end{column}%
      \hfill%
      \begin{column}{.48\textwidth}
      
        \begin{figure}
        \includegraphics[width=6.4cm,height=4.8cm]{pi2.png}
        \caption{1/log(x)}
        \end{figure}

      \end{column}%
      \end{columns}
    \end{frame}

    \begin{frame}{PI - Example}
      \begin{columns}[T] % align columns
      \begin{column}{.48\textwidth}
      \begin{table}[small]
        \centering
        \caption{The example of Pattern-IDF}
        \resizebox{\linewidth}{!}{% Resize table to fit within \linewidth horizontally
        \begin{tabular}{lll}
        \hline
         $Major Word$ & $Minor Word$ & $PI$ \\ \hline
         中国移动 (China Mobile) & 接通 (connect)   & 0.071725 \\ \hline
         中国移动 & cmcc                                  & 0.067261 \\ \hline
         中国移动 & 资费 (charges)   & 0.062408 \\ \hline
         中国移动 & 营业厅 (business hall
        ) & 0.059949 \\ \hline
         中国移动 & 漫游 (roamimg)   & 0.059234 \\ \hline
         ... & ...  & ...  \\ \hline
         中国移动 & 我 (me)   & 0.028889 \\ \hline
         中国移动 & 是 (be)   & 0.027642 \\ \hline
         中国移动 & 的 (of)   & 0.026346 \\ \hline
        \end{tabular}}
        \end{table}

      
      \end{column}%
      \hfill%
      \begin{column}{.48\textwidth}
      \begin{table}[small]
        \centering
        \caption{The entropy of Pattern-IDF for each Major Word}
        \resizebox{.65\linewidth}{!}{% Resize table to fit within \linewidth horizontally
        \begin{tabular}{ll}
        \hline
         $Major Word$   & H \\ \hline
         眼病  (eye disease)     & 0.889971 \\ \hline
         丰收年 (harvest year)      & 0.988191 \\ \hline
         血浆 (plasma)            & 1.033668 \\ \hline
         脊椎动物 (vertebrate)    &  1.083438 \\ \hline
         水粉画 (gouache painting)&  1.180993 \\ \hline
         ...            & ...        \\ \hline
         现在 (now)     & 9.767768   \\ \hline
         什么 (what)  & 10.219045  \\ \hline
         是 (be)         & 10.934950   \\ \hline
        \end{tabular}}
        \end{table}

        
      \end{column}%

      \end{columns}
      
      \begin{equation}
      \Fontvi
      PI_{norm}(D_i|D_j) = \frac{PI(D_i|D_j)}{\sum_{i=1}^n{PI(D_i|D_j)}}
      \end{equation}
      \begin{equation}
      \Fontvi
      H(D_j) = - \sum_{i=1}^n{PI_{norm}(D_i|D_j)\log_2{PI_{norm}(D_i|D_j)}}
      \end{equation}

    \end{frame}

    \begin{frame}{PI - Ranking}
      For each comment $c$ in candidates, given a query (new post) $q$, we 
      calculate the score by $PI$ as follow:
      \begin{equation}
        Score_{PI}(q, c) = \frac{\sum_{D_j \in q}{\sum_{D_i \in c}{PI(D_i|D_j)}}}{len(c) * len(q)}
      \end{equation}
      Then we define rank score as follow:
      \begin{equation}
        Rank_{PI} = (1 + \frac{Score_{PI}(q, c)}{\max{Score_{PI}(q, c)}}) * Sim_{W2V}(q, c)*Sim_{LSA}(q, c)  
      \end{equation}
    \end{frame}

    \begin{frame}{TextRank + Pattern-IDF}
      In this method, We add each comment sentence in candidates as a vertex in the graph and use sentence Word2Vec similarity as edges between vertices in the graph.

      For $N$ candidates, we construct $ N \times N $ matrix $M$. $M_{ij} = Sim_{w2v}(candidate_i, candidate_j)$. 

      At time $t = 0$, We initiate a N-dimension vector $P$ , here $N$ is the number 
      of comment candidates. And each entry of $P$ is defined as the score of Pattern-IDF between the query (new post) $q$ and corresponding comment $c_i$ in candidates:
      \begin{equation}
        P_i = Score_{PI}(q, c_i)
      \end{equation}
    \end{frame}

    \begin{frame}{TextRank + Pattern-IDF}
      Then we compute iteratively
      \[
      R(t+1) = 
      \begin{bmatrix}
          (1-d)/N       \\
          (1-d)/N       \\
          \hdotsfor{1} \\
          (1-d)/N       
      \end{bmatrix}
      +d
      \begin{bmatrix}
          M_{11} & M_{12} & M_{13} & \dots  & M_{1N} \\
          M_{21} & M_{22} & M_{23} & \dots  & M_{2N} \\
          \vdots & \vdots & \vdots & \ddots & \vdots \\
          M_{N1} & M_{N2} & M_{N3} & \dots  & M_{NN}
      \end{bmatrix}
      R(t)
      \]
      Stop when $|R(t+1)-R(t)|<\epsilon$, $\epsilon = 10^{-7}$

      Finally, we get the score $P_i$ for each comment in candidates. 
    \end{frame}

  \begin{frame}{Experiment}
    \begin{itemize}
      \item{Nders-C-R5: } LDA + Word2Vec + LSTM-Sen2Vec 
      \item{Nders-C-R4: } LSA + Word2Vec + LSTM-Sen2Vec 
      \item{Nders-C-R3: } R4 + TextRank (Words as vertices)
      \item{Nders-C-R2: } R4 + Pattern-IDF 
      \item{Nders-C-R1: } R4 + Pattern-IDF + TextRank (Sentences as vertices)
    \end{itemize}
  \end{frame}

  \begin{frame}{官方结果}
    \begin{table}
\centering
\caption{The official results}
\label{tab:commands}
\begin{minipage}{\columnwidth}
\begin{center}
\begin{tabular}{|l|c|c|c|}
\hline
 Run        &  Mean nG@1  &  Mean P+  &  Mean nERR@10  \\ \hline
 Team 1 & 0.5867 & 0.6670 & 0.7095 \\ \hline
 Team 2 & 0.5080 & 0.6080 & 0.6492 \\ \hline
 Team 3 & 0.4980 & 0.5818 & 0.6105 \\ \hline
 Nders-C-R1 & 0.4593 & 0.5394 & 0.5805 \\ \hline
 Nders-C-R2 & 0.4743 & \textbf{0.5497(5th)} & \textbf{0.5882(5th)} \\ \hline
 Nders-C-R3 & 0.4647 & 0.5317 & 0.5768 \\ \hline
 Nders-C-R4 & \textbf{0.4780(4th)} & 0.5338 & 0.5809 \\ \hline
 Nders-C-R5 & 0.4550 & 0.5495 & 0.5868 \\ \hline
 \textcolor{red}{R2 \ vs. \  R4}  & \textcolor{red}{$\downarrow$0.77\%} & \textcolor{red}{$\uparrow$2.98\%} & \textcolor{red}{$\uparrow$1.26\%} \\ \hline

\end{tabular}
\end{center}
\end{minipage}
\end{table}
  \end{frame}

  \begin{frame}[standout]
    Questions?
  \end{frame}

\end{document}

